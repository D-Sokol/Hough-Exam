% Взаимодействие рентгеновского излучения с веществом. Сведение зарегистрированных данных к виду преобразования Радона.

Рассмотрим монохроматический пучок излучения (в данном случае -- рентгеновского), идущего от далекого точечного источника и проходящего через однородный слой вещества толщиной $D$.
\begin{statement}{Закон Бугера-Ламберта-Бера}
    Интенсивность излучения после прохождения вещества описывается следующей формулой:
    \begin{equation*}
        I = I_0 e^{-\mu_0 D},
    \end{equation*}
    где $I_0$ -- интенсивность излучения перед слоем, $\mu_0$ -- линейный коэффициент поглощения, зависящий от вещества и от длины волны. Пренебрегая рассеянием, принимается, что для вакуума $\mu_0 = 0$.
\end{statement}

Этот закон можно обобщить на случай неоднородного слоя следующим образом:
\begin{equation*}
    I = I_0 e^{-\int_0^D \mu(x) dx}
\end{equation*}

Рассмотрим помещенный в работающий по параллельной схеме томограф объект, обладающий неизвестной функцией поглощения $\mu(x, y)$. Тогда непосредственно регистрируемые томографом данные задаются следующими формулами:
\begin{equation}
\label{10_1}
    I(\theta, s) =
    I_0 e^{-\int_{l_{\theta, s}} \mu(x, y) dl} =
    I_0 \exp\left( -\int_{\mathbb{R}^2} \mu(x, y) \delta(x\cos\theta + y\sin\theta - s) dx dy \right),
\end{equation}
где $I_0$ -- результат, получаемый на детекторе в отсутствие объекта.

При помощи элементарных преобразований данные из \eqref{10_1} сводятся к Радон-образу функции поглощения:
\begin{equation*}
    \ln \frac{I_0}{I(\theta, s)} =
    -\ln \frac{I(\theta, s)}{I_0} =
    \int_{\mathbb{R}^2} \mu(x, y) \delta(x\cos\theta + y\sin\theta - s) dx dy =
    \left[ \mathcal{R}\mu \right](\theta, s).
\end{equation*}

Таким образом, взяв логарифм от ослабления сигнала и применив обратное преобразование, можно вычислить функцию поглощения.
