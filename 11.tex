% Преобразование Радона. Синограмма.

\begin{definition}{Преобразование Радона}
    Пусть $l_{\theta, s}$ -- прямая, направляющий вектор которой направлен под углом $\theta$ ($\theta = 0$ соответствует горизональной прямой) и удаленная от начала координат на расстояние $s$. Тогда преобразованием Радона функции $f(x, y)$ называется интеграл этой функции по параметризованной прямой:
    \begin{gather*}
        \left[ \mathcal{R}f \right](\theta, s) =
        \int_{l_{\theta, s}} f(x, y) dl =
        \int_{\mathbb{R}^2} f(x, y) \delta(x \cos \theta + y \sin \theta - s) dx dy =\\=
        \int_{\mathbb{R}} f(s\cos\theta + z\sin\theta, s\sin\theta - z\cos\theta) dz.
    \end{gather*}
\end{definition}

Каждая точка Радон-образа функции представляет собой <<сумму>> по прямой с определенными параметрами. Как правило, по обоим параметрам рассматривается равномерная дискретная сетка значений, где $\theta$ меняется от $0$ до $\pi$, $s$ -- от $0$ до некоторого максимального значения, соответствующего размеру сцены. Полученный массив значений называется синограммой, так как Радон-образом точечной функции является синусоида.
