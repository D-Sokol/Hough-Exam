% Трехмерное быстрое преобразование Хафа для прямых. Параметризация, описание работы, вычислительная сложность.

Трехемерное быстрое преобразование Хафа для прямых позволяет вычислять сумму по дискретным прямым в изображении размера $N \times N \times N$ вокселей. Рассматриваемые прямые -- преимущественно ориентированные вдоль оси $z$. Прямая параметризуется 4 параметрами и проходит через следующие воксели на верхней и нижней гранях куба:
\begin{equation*}
    (s_1, s_2, 0), \quad
    (s_1 + t_1, s_2 + t_2, 0).
\end{equation*}

При этом считается, что все параметры могут меняться от $0$ до $N-1$. Заметим, что проекции прямой на плоскости $Oxz, Oyz$ представляют собой плоские диадические паттерны (в основном потому что именно так определяется трехмерная диадическая прямая).

Следовательно, для начала необходимо преобразовать изображение в четырехмерное. Исходное изображение располагается в пространстве с нулевой координатой по 4 измерению, все остальные пространства заполняются нулевыми значениями.
По аналогии с двумерным преобразованием, основная стратегия состоит в получении на каждом шаге отдельных <<слоев>>, значения в каждом из которых описывают суммы по всем возможным паттернам данной высоты $2^k$, начинающихся с низа данного слоя.
Но если в двумерном случае каждая группа должна была иметь размер $2^k \times N$, то для данного случая потребуются группы $N \times N \times 2^k \times 2^k$.
При этом первые две координаты практически не меняют своего смысла -- они соответствуют $s_1, s_2$, то есть описывают, в какой точке $x, y$ слоя \texttt{group[:, :, 0, 0]} (или, что то же самое, \texttt{image[:, :, 2**k * groupid, 0]}) начинается паттерн, описываемый вокселем. Третья координата $z$ -- это ось, вдоль которой распределяются и объединяются попарно группы. Четвертое измерение необходимо, чтобы было куда записывать получающиеся значения; при этом на $k$-й итерации используются только первые $2^k$ пространств, а остальные заведомо заполнены нулевыми значениями.

На каждом шаге две соседние группы объединяются в одну, и в двумерный срез с координатами \texttt{group[:, :, 2*i1+r1, 2*i2+r2]} записывается поэлементная сумма \texttt{group[:, :, i1, i2]} и \texttt{group[:, :, 2**k+i1, 2**k+i2]}, причем второй массив предварительно сдвигается на $(i_1+r_1, i_2+r_2)$ элементов по последним координатам. Например, после первой итерации группы будут иметь размер $N \times N \times 2 \times 2$, причем две последние координаты описывают тип паттерна (вертикальный, диагональный по одной координате, диагональный по другой координате или диагональный по обеим координатам).

На каждой итерации обрабатывается $\frac{N}{2^{k+1}}$ групп ($k=0, 1, \dots, \log_2 N - 1$). В каждой группе для всех значений $i_1, i_2, r_1, r_2$, то есть $2^{k} \cdot 2^{k} \cdot 2 \cdot 2 = 2^{2k+2}$ раз суммируются двумерные массивы из $N^2$ элементов. Таким образом, сложность алгоритма составляет
\begin{equation}
\label{complexity-fht-3d-lines}
    \Theta\left( \sum_{k=1}^{\log_2 N - 1} \frac{N}{2^{k+1}} \cdot 2^{2k+2} N^2 \right) =
    \Theta\left( 2N^3 + 4N^3 + 8N^3 + \dots + N^4 \right) =
    \Theta\left( N^4 \right).
\end{equation}

Наивная реализация потребовала бы $O(N^5)$ операций. Можно сказать, что трехмерное БПХ для прямых вычисляет сумму по каждой прямой в среднем за константное время.


% Предста8ьте куб, стоящий на столе на одной грани.
% Аккуратно по8ерните его на 45 градусо8 так, чтобы он стоял на одном ребре.
% Теперь по8ернте его еще раз на 45 градусо8 так, чтобы он стоял на одной 8ершине.
% За д8а шага получился куб, стоящий на столе одной точкой.

% За три шага аналогичную операцию можно проделать с тессерактом.
% Для простоты можно пропустить пер8ые д8а шага, 8зя8 куб, стоящий на 8ершине и распространи8 его на некоторое расстояние 8глубь по чет8ертому изменению. Осталось по8ернуть тессеракт 8сего один раз.

