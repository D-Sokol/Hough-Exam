% Теорема о центральном сечении.

Преобразование Фурье функций от одной и от двух переменных задаются следующими формулами:
\begin{gather}
\label{fourier1D}
    \hat{f}(\omega) = \int_{\mathbb{R}} f(t) e^{-2\pi i (\omega t)} dt,\\
\label{fourier2D}
    \hat{F}(u, v) = \int_{\mathbb{R}^2} f(x, y) e^{-2\pi i (xu + yv)} dx dy.
\end{gather}

Также введем сокращенное обозначение для преобразования Радона, считая $\theta$ фиксированным параметром, а $s$ -- переменной:
\begin{equation*}
    p_\theta(s) := \left[ \mathcal{R}f \right]\left( \theta, s \right).
\end{equation*}

\begin{theorem}{О центральном сечении.}
\label{fourier_slice_thm}
    Преобразование Фурье от $p_\theta(s)$ совпадает со значениями двумерного преобразования Фурье от $f(x, y)$ на некоторой прямой:
    \begin{equation*}
        \hat{p}_\theta(\omega) = F(\omega\cos\theta, \omega\sin\theta)
    \end{equation*}
\end{theorem}
\begin{proof}
    Преобразуем определение преобразований Фурье и Радона, используя основное свойство дельта-функции (а также $\delta(t) = \delta(-t)$):
    \begin{gather*}
        \hat{p}_\theta(\omega) =
        \int_{\mathbb{R}} p_\theta(s) e^{-2\pi i \omega s} ds =\\=
        \int_{\mathbb{R}^3} f(x, y) \cdot \delta(x\cos\theta + y\sin\theta - s) e^{-2\pi i \omega s} dx dy ds =
        \int_{\mathbb{R}^2} f(x, y) e^{-2\pi i \omega \left( x\cos\theta + y\sin\theta \right)} dx dy;\\
        F(u=\omega\cos\theta, v=\omega\sin\theta) =
        \left.\int_{\mathbb{R}^2} f(x, y) e^{-2\pi i \left( x u + y v \right)}\right|_{\substack{u=\omega\cos\theta\\v=\omega\sin\theta}} =
        \hat{p}_\theta(s).
    \end{gather*}
\end{proof}

Таким образом, прямая, вырезанная из двумерного Фурье-образа исходной функции, проходящая через начало координат, фактически описывает интегралы этой функции вдоль всех прямых, параллельных вырезанной. Получить их можно при помощи обратного к \eqref{fourier1D} преобразования.
