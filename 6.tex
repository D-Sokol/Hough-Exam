% Преобразование Хафа и быстрое преобразование Хафа. Описание работы алгоритмов и их вычислительных характеристик.

% Радон: прямые, не обязательно дискретные.
% Хаф: дискретные объекты, не обязательно прямые.
% Дискретный Радон == Хаф для прямых.

В общем случае преобразование Хафа ставит в соответствие каждому паттерну из заданного семейства сумму значений пикселей изображения, принадлежащих этому паттерну. В данном случае рассматривается преобразование Хафа для дискретных прямых на двумерном изображении, которое также может быть названо дискретным преобразованием Радона.

Быстрое преобразование Хафа -- это вариация алгоритма, позволяющая понизить сложность алгоритма с $\Theta(n^3)$ до $\Theta(n^2 \log n)$ за счет использования диадических паттернов для приближения прямых.

Диадические паттерны описывают преимущественно вертиальные прямые с наклоном вправо, то есть в каждой строке $i=\const$ содержат ровно один пиксель и $j(i)$ нестрого возрастает. Они задаются следующим образом:
\begin{enumerate}
\item
    Существует один паттерн высоты 1, состоящий из одного пикселя
\item
    Существует $2^n$ паттернов высоты $2^n$ (или порядка $n$), $n > 0$. Паттерн под номером $i$ (нумерация начинается с нуля) состоит из двух состыкованных по вертикали паттернов высоты $2^{n-1}$ под номером $\left\lfloor \frac{i}{2} \right\rfloor$.
    Если $i$ четно, то нижний пиксель верхнего паттерна располагается над верхним пикселем нижнего паттерна.
    Если $i$ нечетно, то верхний паттерн дополнительно сдвигается на один пиксель вправо.
\end{enumerate}

В силу построения диадические паттерны содержит множество пересечений. В частности, любой паттерн высоты больше 1 делит как верхнюю, так и нижнюю половину с другим паттерном той же высоты. Поэтому для вычисления сумм по всем $n^2$ паттернам некоторого порядка достаточно вычислить $n^2$ сумм по паттернам меньшего порядка, а не $2n^2$, как было бы для наивной реализации.

Рассмотрим алгоритм БПХ более подробно. Пусть имеется квадратное изображение со стороной $N = 2^n$. Изначальное изображение можно рассматривать как набор сумм по паттернам порядка $0$. Чтобы получить в том же изображении набор сумм по паттернам порядка $1$, проделаем следующую операцию:
\begin{itemize}
\item
    Разделим изображение на $2^{n-0-1}$ групп, каждая из которых содержит $2^{0+1}$ последовательных строк.
\item
    Каждую группу разделим на верхнюю и нижнюю половины. Затем во все строки группы с номером $2i+r$ нужно записать поэлементную сумму $i$-х строк верхней и нижней половины, сместив строку из верхней половины вправо на $i+r$ элементов, где $i$ меняется от $0$ до $2^{0}$ невключительно (то есть, на начальной итерации $i=0$), а $r \in \{0, 1\}$ -- остаток от деления на 2.
\item
    В силу определения диадических паттернов, после предыдущего шага в $i$-й строке каждый группы записаны суммы по диадическим паттернам $1$ порядка с номером $i$, причем все они начинаются с различных элементов нижней строки группы.
\end{itemize}

Так как полученное в конце условие очень похоже на изначальное (набор сумм по паттернам порядка $1$), то неудивительно, что продолжная аналогичную операцию с соответственно увеличивающимися размерами групп после $n$ шагов мы получим массив, содержащий суммы по паттернам высоты $2^n = N$, начинающимся с $0$ строки изображения. Это и есть суммы по диадическим прямым определенного наклона. Для получения суммы по всем прямым, проходящим через изображение, следует воспользоваться БПХ для транспонированных/отраженных/... копий изображения и объединить полученные результаты.

Оценим сложность алгоритма быстрого преобразования Хафа. На каждой итерации значения каждого пикселя обновляются как сумма двух значений. При этом выполняется ровно $n$ итераций. Следовательно, итоговая сложность составляет $\Theta(N^2 \cdot n) = \Theta(N^2 \log N)$, где $N$ -- сторона изображения.
