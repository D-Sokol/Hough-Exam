% Сочетание БПХ и принципа четырех русских для случаях прямых в трехмерном пространстве.

Как было показано в \eqref{complexity-fht-3d-lines}, сложность быстрого трехмерного преобразования Хафа для прямых составляет $\Theta(N^4)$ и вычисляет суммы по всем диадическим дискретным прямым. Но если требуется найти сумму не по всем прямым, а по $O(N^3)$ выбранных, асимптотическая сложность наивного суммирования совпадает со сложностью ТБПХ.

\textbf{Алгоритм четырех русских} -- это способ ускорить вычисление сумм по $O(N^3)$ прямых за счет совмещения обоих описанных алгоритмов. В нем предлагается выполнить не все $n = \log_2 N$ шагов ТБПХ, а только $x$ первых итераций (способ выбора $x$ описан ниже), после чего для каждой из прямых непосредственно просуммировать $\frac{N}{2^x}$ значений, каждое из которых хранит сумму по части прямой высоты $2^x$. Оценим сложность обоих шагов алгоритма:
\begin{enumerate}
\item
    Сложность выполнения первых $x$ итераций ТБПХ:
    \begin{equation*}
        \Theta\left( 2N^3 + 4N^3 + \dots + 2^x N^3 \right) =
        \Theta\left( N^3 2^{x} \right).
    \end{equation*}
\item
    Сложность суммирования по прямым:
    \begin{equation*}
        O(N^3) \cdot \Theta\left( \frac{N}{2^x} \right) =
        O\left( N^4 2^{-x} \right)
    \end{equation*}
\end{enumerate}

Очевидно, что асимптотический минимум получается при $2^x = \sqrt{N}$, поскольку в противном случае одно из слагаемых будет иметь худшую асимптотику. Таким образом, необходимо сделать $\frac{1}{2} \log_2 N$ шагов алгоритма ТБПХ, после чего перейти к суммированию по прямым, и итоговая сложность алгоритма составит $O\left( N^{3.5} \right)$.
Необходимо отметить, что рассуждения приводятся для случая, когда количество прямых составляет $\Theta(N^3)$, в противном случае оптимальное количество шагов и итоговая сложность могут оказаться другими.
