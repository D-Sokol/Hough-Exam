% Трехмерное быстрое преобразование Хафа для плоскостей. Параметризация, описание работы, вычислительная сложность.

Трехмерное быстрое преобразование Хафа для плоскостей -- это алгоритм, позволяющий быстро подсчитать сумму по всем плоскостям с определенным наклоном в изображении размера $N \times N \times N$. При этом плоскость рассматривается как объект, составленный из диадических паттернов и проходящий через точки со следующими координатами (для удобства приведены четыре точки, хотя для задания плоскости достаточно любых трех):
\begin{gather}
\label{7_1}
    (s, 0, 0), \quad
    (s + t_1, 0, N-1), \quad
    (s + t_2, N-1, 0), \quad
    (s + t_1 + t_2, N-1, N-1), \\
\nonumber
    0 \le s \le N-1,
    0 \le t_1 \le N-1,
    0 \le t_2 \le N-1.
\end{gather}

При помощи наивного алгоритма Хафа можно вычислить суммы по $N^3$ плоскостей, содержащих $O(N^2)$ пикселей, за $O(N^5)$ операций. Как и в двумерном случае, эту асимптотику можно улучшить.

Рассмотрим плоскость, образованную диадическими паттернами в соответствии с \eqref{7_1}. В каждом сечении $z = \const$ образ этой плоскости представляет собой диадический паттерн с $\hat{t} = t_2, \hat{s} = s + I_{t_1}(z)$, где $I_{t}(k)$ -- смещение пикселя в строке $k$ в паттерне со склонением $t$. Нам требуется вычислить суммы по всем пикселям в объединении этих паттернов. В качестве первого шага можно вычислить суммы по каждому из этих диадических паттернов в отдельности.

Если в исходном изображении применить к каждому массиву в сечении по оси $Z$ двумерное быстрое преобразование Хафа, то смысл осей изменится: теперь координаты в массиве будут представлять не $(x, y, z)$, а $(\hat{s}, \hat{t}, z)$. Таким образом, искомая сумма преобразуется в сумму по пикселям $\left\{ \left( s + I_{t_1}(z), t_2, z \right)_{z=1}^{N-1} \right\}$, что в свою очередь является диадическим паттерном в плоскости $\hat{t} = \const$.

Применив к каждому массиву в сечении по оси $\hat{T}$ двумерное быстрое преобразование Хафа, в точке $\left( s, t_2, t_1 \right)$ получаем искомую сумму.

Поскольку в ходе работы алгоритма мы вычислили $2N$ двумерных БПХ, общая сложность алгоритма составляет $\Theta\left( N^3 \log N \right)$. Можно сказать, что за счет переиспользования вычисленных сумм сумма по каждой плоскости вычисляется за $O(\log N)$ в среднем.
% Хотя я даже не представляю, в какой ситуации это может быть важно.
