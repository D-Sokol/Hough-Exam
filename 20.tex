% Быстрая линейная бинарная кластеризация с помощью БПХ.

Рассмотрим некоторое изображение в градациях серого с темным фоном, светлые точки которого нужно разделить одной гиперплоскостью на два кластера наиболее <<естественным>> способом. Для простоты будем считать, что изображение двумерно и имеет размер $n \times n$, хотя данный алгоритм потенциально может быть использован для изображений любой размерности, в таком случае необходимо найти разделяющую прямую.

Будем обозначать множества, на которые прямая разделяет множество всех точек изображения $\Omega$ как $A, B$. Считая, что яркость точки задается как функция от координат $P\left( \vec x \right)$, введем следующие характеристики множеств:
\begin{gather*}
    \omega(X) = D^0(X) = \sum_{\vec x \in X} P\left( \vec x \right), \\
    D^1_i(X) = \sum_{\vec x \in X} x_i P\left( \vec x \right) = \left( \sum_{\vec x \in X} \vec x P\left( \vec x \right) \right)_i, \\
    D^2_{ij}(X) = \sum_{\vec x \in X} x_i x_j P\left( \vec x \right) = \left( \sum_{\vec x \in X} \vec x \otimes \vec x P\left( \vec x \right) \right)_{ij}. \\
\end{gather*}
Следует отметить, что, во-первых, все приведенные статистики являются аддитивными ($D^p\left( X \cup Y \right) = D^p(X) + D^p(Y)$ для любых непересекающихся $X, Y$), а во-вторых, матрица $D^2$ симметрична, что снижает количество элементов, которые нужно вычислять.

В качестве критерия оптимизации предлагается выбрать одну из следующих целевых функций, которые могут быть выражены через аддитивные статистики:
\begin{gather*}
    \omega(A) \tr\sigma(A) + \omega(B) \tr\sigma(B) \to \min_{A, B},\\
    \omega(A) \lambda_2(A) + \omega(B) \lambda_2(B) \to \min_{A, B},\\
\end{gather*}
где $\sigma(A)$ -- внутриклассовая матрица ковариации, $\lambda_2$ -- ее второе собственное значение.

Поскольку данные функционалы невыпуклые, единственным надежным способом выбора прямой является полный перебор всех прямых из некоторого ограниченого семейства. Для эффективного вычисления массива значений функционала нужно вычислить 6 (для двумерного изображения; в общем случае по количеству независимых значений среди всех $D^p$ для произвольного фиксированного множества) массивов значений аддитивных статистик, вычисленных для всех полуплоскостей. Эта операция может быть выполнена при помощи быстрого преобразования Хафа.

Обозначим вычисляемую статистику как $T\left( X \right) \in \left\{ D^0, D^1_1, D^1_2, D^2_{11}, D^2_{12}, D^2_{22} \right\}$. За $O\left( n^2 \right)$ операций можно вычислить массив того же размера, что и изображение, где каждый элемент равен значению $T$ на множестве из одного соответствующего пикселя.

Выполним к данному массиву операцию кумулятивного суммирования вдоль оси $Oy$. После этого каждый элемент по определению операции содержит сумму значений $T$ по всем пикселям непосредственно под ним (включая сам пиксель), а сумма по полуплоскости, лежащей под прямой, заданной параметрами $(s, t)$ теперь может быть вычислена как сумма всех элементов этой прямой. Применив к массиву быстрое двумерное преобразование Хафа, получим массив, где элемент $I_{st}$ содержит статистику $T$, вычисленную по полуплоскости, лежащей ниже прямой с соответствующими параметрами. Статистика по второй полуплоскости может быть вычислена при помощи вычитания найденного значения из $T\left( \Omega \right) = \const$. После вычисления всех статистик при помощи поэлементных операций над двумерными массивами вычисляются значения целевых функций для каждой прямой. Среди полученных значений находится минимум. Прямая, соответствующая положению этого минимума, является оптимальной разделяющей прямой.

Итоговая сложность алгоритма составляет $O\left( m n^2 \log n \right)$, где $m$ -- количество нужных статистик (для описанных критериев $m = 6 = \const$), так как именно столько операций требуется на вычисление всех линограмм, а все остальные операции выполняются за $O\left( n^2 \right)$.
