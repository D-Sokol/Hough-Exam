% Преобразование Радона. Дискретное преобразование Радона. Оценка сложности.

\begin{definition}{Преобразование Радона}
    Пусть $l_{\theta, s}$ -- прямая, направляющий вектор которой направлен под углом $\theta$ ($\theta = 0$ соответствует горизональной прямой) и удаленная от начала координат на расстояние $s$. Тогда преобразованием Радона функции $f(x, y)$ называется интеграл этой функции по параметризованной прямой:
    \begin{gather*}
        \left[ \mathcal{R}f \right](\theta, s) =
        \int_{l_{\theta, s}} f(x, y) dl =
        \int_{\mathbb{R}^2} f(x, y) \delta(x \cos \theta + y \sin \theta - s) dx dy =\\=
        \int_{\mathbb{R}} f(s\cos\theta + z\sin\theta, s\sin\theta - z\cos\theta) dz.
    \end{gather*}
\end{definition}

Рассмотрим теперь дискретную версию $f(x, y)$, заданную, например, как среднее по квадратной области:
\begin{equation*}
    \hat{f}(i, j) =
    \frac{1}{h^2} \int_{ih - \frac{h}{2}}^{ih + \frac{h}{2}} \int_{jh - \frac{h}{2}}^{jh + \frac{h}{2}} f(x, y) dy dx.
\end{equation*}

Для определения дискретного преобразования Радона также необходим некоторый алгоритм дискретизации прямой $\Omega\left( s, \alpha \right)$, возвращающий множество координат пикселей $\left\{ \left( i_k, j_k \right)_{k=1}^{m} \right\}$, приближающих непрерывную прямую с соответствующими параметрами.

Тогда \textbf{дискретное преобразование Радона} определяется следующим образом:
\begin{equation*}
    \left[ \hat{\mathcal{R}} \hat{f} \right](\alpha, s) = \sum_{(i,j) \in \Omega(\alpha, s)} \hat{f}\left( i, j \right).
\end{equation*}

Однако такое определение обладает проблемой, связанной с неравномерностью приближения длины прямой количеством пикселей. Например, вертикальная непрерывная прямая, проходящая через центр квадрата со стороной $N$ имеет длину пересечения $N$, а диагональная -- $N\sqrt{2}$. Но обе дискретные версии будут иметь в пересечении с квадратом $N$ пикселей.


Оценим сложность преобразования Радона наивным способом. В разумной параметризации прямых количество дискретных прямых, которые проходят через изображение, составляет $O(n^2)$. Нужно вычислить и сохранить для каждой из них сумму по содержащимся в ней пикселям, количество которых $O(n)$. Считая, что $\Omega$ также обладает не более чем линейной сложностью по $n$, получаем, что для вычисления сумм требуется $O(n^3)$ операций и $O(n^2)$ памяти.

В случае черно-белых (бинарных) преимущественно черных изображений возможно изменить алгоритм: достаточно перебрать все белые точки и для каждой из них прибавлять единицу к сумме по всем прямым, проходящим через эту точку. Таким образом сложность понижается до $O(Cn^2)$, где $C$ -- количество белых точек на изображении. Однако необходимо учесть, что для этой вариации требуется уметь находить прямые, проходящие через заданную точку не более чем за $O(n^2)$ операций, что может быть затруднительно для некоторых видов параметризации.
