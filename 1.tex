% Изображения как структура данных. Базовые операции над изображением. Свертка изображения

Изображения в целом разделяются на растровые и векторные. В данном случае рассматриваются только растровые изображения, загруженные полностью, то есть избавленные от сжатия.

Изображение как структура данных обычно состоит из двух частей:
\begin{enumerate}
\item
    Линейный массив, где последовательно записаны значения всех компонент последовательно для каждого пикселя. В случае цветового пространства RGB это три числа, представляющих собой значения интенсивности соответствующего канала для рассматриваемой точки.
\item
    Заголовок -- небольшая структура постоянного размера. В этой структуре хранится информация о размерах изображения, о том, смещению на сколько байт в памяти соответствует смещение на один пиксель по каждому из направлений массива, а также ссылка на начало массива, соответствующее первому пикселю и служебные данные, необходимые для управления владением массивом.
\end{enumerate}

За счет хранения в заголовке информации о размерах и сдвигах можно быстро получать произвольные срезы изображения (например, выбор произвольной прямоугольной области на изображении), в том числе разреженные (можно взять каждый второй пиксель по одной из координат и каждый третий по другой) или отраженные (поскольку сдвиг может быть отрицательным). Полученные структуры будут использовать тот же самый массив, что и исходное изображение, то есть будет скопирован только заголовок, но не сами данные. При этом сдвиги (stride) и размеры массива могут не соответствовать друг другу, то есть с точки зрения полученного среза вполне возможно, что описываемое изображение будет храниться в памяти с разрывами.

% Здесь еще что-нибудь про базовые операции должно быть.

\begin{definition}{Свертка.}
    Пусть задано небольшое одноканальное небольшое изображение, называемое \textbf{ядром свертки}, размерность которого совпадает с размерностью изображения. Также у ядра свертки должен быть выделен один элемент, называемый \textbf{якорем}. Чаще всего используют ядра размера $3 \times 3$ с якорем в центре, такие, что сумма всех значений ядра равна 1 или 0. \textbf{Сверткой} изображения с ядром той же размерности называется операция преобразования изображения, при которой новое значение каждого пикселя вычисляется следующим образом:
    \begin{enumerate}
    \item
        Ядро совмещается с изображением таким образом, чтобы якорь ядра был совмещен с рассматриваемым пикселем.
        % Вроде бы с математической точки зрения ядро еще надо бы по всем координатам отразить. Впрочем, ядра часто симметричные, да и вообще всем как всегда.
    \item
        Вычисляются все попарные произведения значений пикселей ядра и значений совмещенных с ними значений ядра. При этом существует несколько различных стратегий, какие значения следует принять для пикселей, выходящих за границы изображения: константные значения, отражение изображения относительно границы и т.~д.
    \item
        Сумма этих произведений считается значением соответствующего пикселя свертки.
    \end{enumerate}
\end{definition}

